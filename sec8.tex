\section{Programmazione Generica}
\label{sec:genericity}

L'ultima caratteristica di \Scala che esploreremo in questo tutorial \`e la programmazione generica. Gli sviluppatori \Java dovrebbero essere bene informati dei problemi relativi alla mancanza della programmazione generica nel loro linguaggio, un'imperfezione risolta in \Java 1.5.

La programmazione generica riguarda la capacit\`a   di scrivere codice parametrizzato dai tipi.  Per esempio, un programmatore che scrive una libreria per le liste concatenate pu\`o incontrare il problema di decidere quale tipo dare agli elementi della lista. Dato che questa lista \`e stata concepita per essere usata in contesti differenti, non \`e possibile decidere  che il tipo degli elementi deve essere, per esempio, Int. Questo potrebbe essere completamente arbitrario ed eccessivamente restrittivo.

Gli sviluppatori \Java hanno fatto ricorso all'uso di \code{Object}, che \`e il super-tipo di tutti gli oggetti. Questa soluzione \`e in ogni caso ben lontana dall'esser ideale, poich\'e non funziona per i tipi base (\code{int}, \code{long}, \code{float}, etc.) ed implica che molto type casts dinamico deve esser fatto dal programmatore.

\Scala rende possibile la definizione delle classi generiche (e metodi) per risolvere tale problema. Esaminiamo ci\`o con un esempio del pi\`u semplice container di classe possibile: un riferimento, che pu\`o essere o vuoto o un puntamento ad un oggetto di qualche tipo.
\begin{lstlisting}
class Reference[T] {
  private var contents: T = _

  def set(value: T) { contents = value }
  def get: T = contents
}
\end{lstlisting}
La classe \code{Reference} \`e parametrizzata da un tipo, chiamato \code{T}, che \`e il tipo del suo elemento. Questo tipo \`e usato nel corpo della classe come il tipo della variabile \code{contents}, l'argomento del metodo \code{set}, ed il tipo restituito dal metodo \code{get}.

Il precedente codice d'esempio introduce le variabili in \Scala che non dovrebbero richiedere ulteriori spiegazioni. E' tuttavia interessante notare che il valore iniziale dato a quella variabile \`e \code{_}, che rappresenta un valore di default. Questo valore di default \`e 0 per i tipi numerici, \code{false} per il tipo \code{Boolean}, \code{()} per il tipo \code{Unit} e \code{null} per tutti i tipi oggetto.

Per usare la classe \code{Reference}, \`e necessario specificare quale tipo usare per il tipo parametro \code{T}, che \`e il tipo di elemento contenuto dalla cella. Per esempio, per creare ed usare una cella che contiene un intero, si potrebbe scrivere il seguente codice:
\begin{lstlisting}
object IntegerReference {
  def main(args: Array[String]) {
    val cell = new Reference[Int]
    cell.set(13)
    println("Reference contains the half of " + (cell.get * 2))
  }
}
\end{lstlisting}
Come si pu\`o vedere in questo esempio non \`e necessario fare il \emph{cast} del tipo ritornato dal metodo \code{get} prima di usarlo come intero. Non risulta possibile memorizzare niente di diverso da un intero nella varibile \code{cell}, poich\'e \`e stata dichiarata per memorizzare un intero.